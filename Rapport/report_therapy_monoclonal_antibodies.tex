% !TEX program = xelatex
\documentclass{article}
\usepackage[utf8]{inputenc}
\usepackage[T1]{fontenc}
% \usepackage[english]{babel}
% \usepackage{csquotes}

% with xelatex
\usepackage{fontspec}  % fontspec et xunicode sont facultatifs
\usepackage{xunicode}  % pour les versions postérieures à 2018.

\usepackage[a4paper, left=2.5cm,right=2.5cm,top=2.5cm,bottom=2.5cm]{geometry}
\setlength{\parindent}{1em}
% \setlength{\parskip}{1em}
\usepackage{indentfirst}

\usepackage[sfdefault,lining]{FiraSans} %% option 'sfdefault' activates Fira Sans as the default text font
\usepackage[fakebold]{firamath-otf}
\renewcommand*\oldstylenums[1]{{\firaoldstyle #1}}


\usepackage{graphicx}

\usepackage{biblatex}

\title{
  \textbf{Monoclonal Antibody Therapy}
}
\author{Louis-Justin Tallot}
\date{February 2022}


% \bibliographystyle{plain}
\bibliography{report_therapy_monoclonal_antibodies.bib}

\begin{document}

  \maketitle

  \section*{Introduction}
  Monoclonal antibody therapy is a therapeutic strategy developed in the 1970s
that uses monoclonal antibodies to bind to specific targets such as viruses or tumor cells.
Indeed, monoclonal antibodies -- produced from a single white blood cell clone --
have high specificity to one epitope, and therefore can be used to specifically target
an antigen.

This paper will examine the scientific aspects of monoclonal antibody therapy,
such as the way they are produced and how they act when released as a drug.
The clinical and commercial aspects of this therapeutic way will also be examined.


  \section{What is a monoclonal antibody ?}

    \subsection{What is an antibody ?}

    \subsection{Difference between monoclonal and polyclonal antibodies}

    \subsection{Production of monoclonal antibodies}

    \subsection{Characteristics of a monoclonal antibody}

  \section{How are monoclonal antibodies used for therapy ?}

    \subsection{Therapeutical activity of monoclonal antibodies}

    \subsection{Commercial uses}

    \subsection{Market analysis}
    
  
  \section*{Conclusion}

% \appendix

  \printbibliography

\end{document}
