% !TEX program = xelatex
\documentclass{article}
\usepackage[utf8]{inputenc}
\usepackage[T1]{fontenc}
\usepackage[english]{babel}
\usepackage{csquotes}

\usepackage[a4paper, left=2.5cm,right=2.5cm,top=2.5cm,bottom=2.5cm]{geometry}

\usepackage[sfdefault,lining]{FiraSans} %% option 'sfdefault' activates Fira Sans as the default text font
\usepackage[fakebold]{firamath-otf}
\renewcommand*\oldstylenums[1]{{\firaoldstyle #1}}


\usepackage{graphicx}

\usepackage{biblatex}

\title{
  \textbf{Monoclonal Antibody Therapy}
}
\author{Louis-Justin Tallot}
\date{February 2022}


% \bibliographystyle{plain}
\bibliography{report_therapy_monoclonal_antibodies.bib}

\begin{document}

  \maketitle

  \section*{Introduction}

  \section{What is a monoclonal antibody ?}

    \subsection{What is an antibody ?}

    \subsection{Difference between monoclonal and polyclonal antibodies}

    \subsection{Production of monoclonal antibodies}

    \subsection{Characteristics of a monoclonal antibody}

  \section{How are monoclonal antibodies used for therapy ?}

    \subsection{Therapeutical activity of monoclonal antibodies}

    \subsection{Commercial uses}

    \subsection{Market analysis}
    
  
  \section*{Conclusion}

% \appendix

  \printbibliography

\end{document}
