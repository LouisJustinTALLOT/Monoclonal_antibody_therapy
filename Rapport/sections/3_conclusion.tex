In conclusion, monoclonal antibodies are a promising therapeutic agent for
treatment of cancers and infectious diseases. They are very potent at neutralizing
a specific target and can then be used to eliminate it, either by leveraging the patient's
own immune system capabilities or by bringing a chemical agent attached to the antibody.

The therapeutic market for monoclonal antibodies is growing rapidly, valued
at $160$ billion dollars in 2020 and expected to grow threefold by 2030
\cite{terdale_monoclonal_2021} \cite{insights_monoclonal_2021}.
This growth is accompanied by an important research effort, the research market
being valued at $5,9$ billion dollars in 2020 and expected to grow
at an annual rate of $6 \%$ until 2028 \cite{markets_global_2021}.

Finally, around $500$ monoclonal antibodies have been discovered, leading to
the approval of nearly $100$ mAbs-based drugs. This presents a challenge for
researchers, fighting against viruses and cancers, but more importantly is a great
hope for humankind which can fight its most lethal diseases, even if it does so
at a high cost and thus only the richest countries can access such treatments.