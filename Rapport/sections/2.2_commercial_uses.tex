After reviewing the different mechanisms by which an antibody targeting
an antigen can lead to its elimination by the immune system, we are going
to study commercial products that uses monoclonal antibodies to fight
against tumors, notably.

As it has already been noted, the mAbs serve as a precise means of targeting
specifically a cell or agent that will be treated naturally by the immune system ; 
the mAbs do not suppress any antigen on their own.

\subsubsection{Trastuzumab against HER2-positive breast cancer}

\begin{figure}[H]
    \begin{minipage}{0.495\textwidth}
            \centering
            \includegraphics[width=\textwidth]{../Images/herceptin.jpg}
            \caption{Modelization of the trastuzumab molecule}
            \label{fig:trastuzumab}
    \end{minipage}\hfill
    \begin{minipage}{0.495\textwidth}
            \centering
            \includegraphics[width=\textwidth]{../Images/trastuzumab.jpg}
            \caption{Trastuzumab is sold by Roche under the name Herceptin
            \cite{roche_herceptin_nodate}}
            \label{fig:herceptin}
    \end{minipage}
\end{figure}

Breast cancer is one of the most common cancers in the world.
It concerns 25\% of cancer cases in women, with around 2 million new cases
each year. This type of cancer has thus been the target of many research efforts.

For a particular type of breast cancers called \emph{HER2 positive}, be it early 
or metastatic, which can be particularly aggressive, multiple treatments can
be considered. Surgery, radiation therapy can be used to remove part of the tumor ;
chemotherapy aims at treating chemically the cancer. However, these treatments are
not always fully effective at removing the tumor, especially for metastatic cancers,
which is were tumor cells travel in the bloodstream and fix themselves in another
part of the body.

Trastuzumab, a humanized monoclonal antibody approved by the FDA in the 
United States in 1998, binds to a transmembrane protein called 
the \emph{HER2 receptor} \cite{zhao_trastuzumab_2021}. This protein is
overexpressed in 15 to 25 percent of breast cancers ; this overexpression
is linked with aggressive forms of the disease \cite{piccart-gebhart_trastuzumab_2005}.

Indeed, the proliferation of this protein at the surface of cells leads to their
uncontrolled multiplication, leading to the formation of a tumor.
Trastuzumab, by binding with the HER2 receptor, prevents its dimerization and
proliferation, and thus stops the growth of the cell.

It can by covalently associated with a cytotoxic called DM1 to form
\emph{trastuzumab emtansine}. This antibody-drug conjugate (ADC) 
is associated with a higher disease-free survival rate
\cite{lambert_ado-trastuzumab_2014}.

Herceptin is delivered by injection over the course of one year.
It costs $17000$ € per patient. Herceptin can also be prescribed for treatment of stomach cancer. 
As this drug is comparatively old, multiple
biosimilars have been developed by concurrent laboratories. This was in the
top-3 selling drugs for Roche Pharma ; it brought a $6,08$ billion dollar
revenue in 2019 alone to the lab \cite{fierce_pharma_herceptin_2020}.

