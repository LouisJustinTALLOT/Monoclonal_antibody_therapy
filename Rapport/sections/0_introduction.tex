Monoclonal antibody therapy is a therapeutic strategy developed in the 1970s
that uses monoclonal antibodies to bind to specific targets such as tumoral cells.
Indeed, monoclonal antibodies -- antibodies produced from a single white blood cell --
have high specificity to one epitope, and therefore can be used to specifically target
an antigen.

This paper will examine the scientific aspects of monoclonal antibody therapy
such as the way they are produced and how they act when released as a drug;
The clinical and commercial aspects of this therapeutic way will also be examined.
