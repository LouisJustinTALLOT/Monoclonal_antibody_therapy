% !TEX program = xelatex
\documentclass{beamer}
% \usepackage[utf8]{inputenc}
% \usepackage[T1]{fontenc}
% \usepackage[english]{babel}
% \usepackage{csquotes}

% with xelatex
\usepackage{fontspec}  % fontspec et xunicode sont facultatifs
\usepackage{xunicode}  % pour les versions postérieures à 2018.

% \usepackage[a4paper, left=2.5cm,right=2.5cm,top=2.5cm,bottom=2.5cm]{geometry}
\setlength{\parindent}{1em}
% \setlength{\parskip}{1em}
\usepackage{indentfirst}

\usepackage[sfdefault,lining]{FiraSans} %% option 'sfdefault' activates Fira Sans as the default text font
\usepackage[fakebold]{firamath-otf}
\renewcommand*\oldstylenums[1]{{\firaoldstyle #1}}

\usepackage[none]{hyphenat}

\usepackage{graphicx}

\usepackage{biblatex}

\title{
  \textbf{Monoclonal Antibody Therapy}
}
\author{Louis-Justin Tallot}
\date{February 2022}

\titlegraphic{\centering\vspace{7cm}\includegraphics[width=0.3\textwidth]{../Images/Logo_Mines_ParisTech.png}}


% \bibliographystyle{plain}
\bibliography{report_therapy_monoclonal_antibodies}

\begin{document}
  \maketitle
  \end{frame}



  \section*{Introduction}
  \begin{frame}{Introduction}
    % champ booming, immunothérapie....
    \begin{figure}
        \centering
        \includegraphics[width=0.8\textwidth]{../Images/mab_covid.png}
        % \caption{Neutralization of a virus by antibodies}
        \label{fig:mab_covid}
    \end{figure}

\end{frame}

  \section{What is a monoclonal antibody ?}

    \subsection{What is an antibody ?}
    First of all, let us examine what an antibody is, and see how this
type of molecules possesses remarkable properties that make it suitable
for therapeutic use.

An antibody (also known as Immunoglobulin) is a protein produced by the 
body's immune system that binds to a specific antigen. It is composed of four
polypeptide chains: two identical heavy chains 
and two identical light chains \cite{davies_antibody_1993}. 

The way this chains are self-assembled gives the protein a Y-shaped structure. 
Each of them possesses a terminal high-variability domain, which when grouped spatially
together forms a binding site that can be adapted to a wide variety of antigens.

\begin{figure}[!h]
    \begin{minipage}{0.495\textwidth}
        \centering
        \includegraphics[width=\textwidth]{../Images/schematics_antibody.png}
        \caption{Schematic representation of an antibody} 
        \label{fig:schematics_antibody}
    \end{minipage}\hfill
    \begin{minipage}{0.495\textwidth}
        \centering
        \includegraphics[width=\textwidth]{../Images/immunoglobulin_3D_model.jpg}   
        \caption{3D model of an antibody}
        \label{fig:immunoglobulin_3D_model}
    \end{minipage}
\end{figure}


The variation between immunoglobulins in the low-variability domain
creates multiple antibody classes, also known as \emph{isotypes} :
IgA, IgD, IgE, IgG, or IgM. They all have different functions, but act
following the same general principles that will be discussed later in this paper.


The most important one are :
\begin{itemize}
    \item IgG, which represents $75\%$ of serum antibodies
    in humans. It is involved in the immunity transmitted by a mother
    to her newborn, protecting the child for the first six months of life
    before it can acquire its own immune memory.
    \item IgM, which auto-assembles to form a pentamer, is the largest
    antibody and one of the first to response to an antigen intrusion.
\end{itemize}





    \subsection{Difference between monoclonal and polyclonal antibodies}
    \begin{frame}{Monoclonal \textit{versus} polyclonal antibodies}
    \begin{minipage}{0.5\textwidth}
        \begin{block}{Monoclonal antibody}

            \begin{itemize}
                \item Single antibody species
                \item Binds to a unique specific site
                \item Expensive to produce
            \end{itemize}
        \end{block}

        \begin{block}{Polyclonal antibody}
            \begin{itemize}
                \item Multiple antibody species
                \item Binds to multiple, less specific sites
                \item Cheap to produce
            \end{itemize}
        \end{block}
    \end{minipage}\hfill
    \begin{minipage}{0.45\textwidth}
        \begin{figure}
            \centering
            \includegraphics[width=\textwidth]{../Images/schematics_mono_poly.png}
            \caption{Comparison between monoclonal and polyclonal antibodies}
        \end{figure}    
    \end{minipage}
\end{frame}

    \subsection{Production of monoclonal antibodies}
    We will now examine the details of monoclonal antibody production.
It is done in multiple steps, as summarized on figure \ref{fig:Monoclonal_Antibody_Production}.

\begin{figure}[H]
    \begin{center}
        \includegraphics[width=0.4\textwidth]{../Images/mab_hybridomas.png}
        \caption{The monoclonal antibody production process}
        \label{fig:Monoclonal_Antibody_Production}
    \end{center}
\end{figure}


\subsubsection{Step 1 : Immunization}

In order to begin the production of monoclonal antibodies, it is first
needed to immunize an animal with the antigen of interest. Typically, 
mice are used for this process \cite{leenaars_critical_2005}. 
An \emph{immunogen}, \textit{i.e} an antigen due to induce an immune response
in the animal, is then injected into the animal, along with an adjuvant
aiming at enhancing the immune response.

\begin{figure}[H]
    \begin{minipage}{0.59\textwidth}
        \centering
        \includegraphics[width=\textwidth]{../Images/lymphocyte_b.png}
        \caption{The activation cycle of a B lymphocyte}
        \label{fig:B-cell_activation}
    \end{minipage}\hfill
    \begin{minipage}{0.39\textwidth}
        \centering
        \includegraphics[width=\textwidth]{../Images/B-cell_receptor.jpg}   
        \caption{Schematics of a B-cell receptor}
        \label{fig:B-cell_receptor}
    \end{minipage}
\end{figure}

The animal's blood contains B lymphocytes possessing a transmembrane protein
called \emph{B-cell receptors} (BCR) \cite{seifert_human_2016} 
which is analogous to an immunoglobulin 
attached to the cell's surface, as shown on figure \ref{fig:B-cell_receptor}.
These BCR are of wide diversity, each one of the $10^{11}$ B-cells of the human immune system 
being capable of binding to a different epitope \cite{reth_chapter_2015}.
It should be specified that each mature B-cell have around $120 000$ BCR,
all identical to each other \cite{reth_chapter_2015}.
When a BCR encounters an antigen that is it capable of binding
with, it is said to be \emph{activated}. The cell is then transformed in a 
lymphoblast which evolves in two type of cells : \emph{memory B-cells}
which helps the immune system remembering previous infections, and 
\emph{plasma cells}, which will produce the antibodies specific to the antigen that
was recognized by the BCR \cite{levine_b-cell_2000}.


\subsubsection{Step 2 : Fusion and selection}

Once the animal has had sufficient time to develop a good immune
response to the antigen it was injected with, its splenic B-cells
that produce antibodies specific to the antigen are collected.

On their own, these B-cells coming from the animal's spleen produce
monoclonal antibodies ; however, we cannot stop the process there
as these cells have a limited life expectancy and thus cannot be
use in an industrial process without having to restart the immunization
stage frequently with new animals.

Thus, a technique called \emph{fusion}, invented by G. Köhler and
C. Milstein in 1975 \cite{kohler_continuous_1975} is used to immortalize
the B-cells of interest. In order to do so, these cells are fused with
compatible myeloma cells, which are cancerous cells -- and thus immortal.

It should be noted that as cells have two ways of producing naturally
nucleotides : by the \emph{de novo} pathway, and by a salvage way using the enzyme
hypoxanthine guanine phosphoribosyl transferase (HGPRT) \cite{mckeran_use_1976}.
HGPRT negative myeloma cells are used for the fusion process ; they only rely
on the \emph{de novo} pathway to produce nucleotides.
% By using HGPRT negative myeloma cells, it is ensured that they do not produce
% neither antibodies, nor nucleotides as they are unable to produce purines
% such as adenine and guanine. 

When fused with (mortal) B-cells -- that are HGPRT positive --
in polyethylene glycol -- which causes the cells to fuse together --,
the resulting cells are immortal, and produce the desired antibodies.

Some selection is then necessary to remove the unfused cells. In order
to do so, a selective medium containing hypoxanthine, aminopterin, and 
thymidine, otherwise known as "HAT", is added \cite{nelson_monoclonal_2000}. 
The aminopterin blocks the \emph{de novo} pathway, which is the only one 
available to HGPRT negative cells. As a result, the unfused myeloma cells die,
whereas the hybridomas resulting of the fusion process survive, as the
HGPRT enzyme is brought by the splenic B-cell.

At the end of this step is thus obtained a clone of B-cells that reproduce
continuously and secrete antibodies that are specific to the antigen
initially used to immunize the animals.

\subsubsection{Step 3 : Screening}

Once the fusion and selection step has been carried out, it is necessary
to screen the resulting hybridomas. Indeed, we want to ensure that we
only produce the desired monoclonal antibodies to achieve optimum
specificity.

As all hybridomas do not grow at the same rate, the selection process
should be done daily for a duration of multiple weeks \cite{nelson_monoclonal_2000}.

A common technique involve an Epstein-Barr viral associated protein or peptide
coated on to plastic ELISA plates. Along with a chromogenic substrate and a
secondary enzyme, a colored product is obtained
for positive hybridomas \cite{grunow_cell_1994} \cite{nelson_monoclonal_2000}.

Some research has been done on this step to improve the technique and
speed it up, for example by detecting the antigen-antibody reaction by the 
increase of light scatter on the surface of indium metal \cite{rej_screening_1988}.

Afterwards, the selected hybridomas can then be placed on a culture medium
and be grown in culture flasks.

\subsubsection{Step 4 : Characterization}

Isotype determination is then performed along with isolation and sub-culture
of each cell, to ensure that only one class of antibody is obtained,
and that the clone is coming from only one population 
of fused B-cells \cite{nelson_monoclonal_2000}.

\subsubsection{Step 5 : Production}

Once all these steps have been followed, we can harvest monoclonal
antibodies in bulk, using for example surface expanded culture flasks
\cite{nelson_monoclonal_2000}. It should be noted that quality controls
should be performed regularly, to ensure that the antibodies are the ones
we want to produce, and that there have not been any reaction with other
antigens, nor that the antibodies exhibit dual 
specificity \cite{nelson_monoclonal_2000}.

Indeed, the final product should aim at limiting treatment-related
allergic responses that could be triggered by foreign antigen exposure
\cite{national_research_council_us_committee_on_methods_of_producing_monoclonal_antibodies_large-scale_1999}.
side effects of the drug.

Finally, some research has been done to transform murine antibodies
(those derived from mice splenic B-cells) into chimeric or even humanized antibodies,
that are better accepted by the patient immune system, thus limiting
unwanted side effects \cite{noauthor_how_2020} \cite{ahmadzadeh_antibody_2014}.


    \subsection{Characteristics of a monoclonal antibody}
    \input{sections/1.4_characteristics_monoclonal_antibody.tex}

  \section{How are monoclonal antibodies used for therapy ?}

    \subsection{Therapeutic activity of monoclonal antibodies}
    \input{sections/2.1_therapeutic_activity.tex}

    \subsection{Commercial uses}
    After reviewing the different mechanisms by which an antibody targeting
an antigen can lead to its elimination by the immune system, we are going
to study commercial products that uses monoclonal antibodies to fight
against tumors, notably.

As it has already been noted, the mAbs serve as a precise means of targeting
specifically a cell or agent that will be treated naturally by the immune system ; 
the mAbs do not suppress any antigen on their own.

\subsubsection{Trastuzumab against HER2-positive breast cancer}

\begin{figure}[H]
    \begin{minipage}{0.495\textwidth}
            \centering
            \includegraphics[width=\textwidth]{../Images/herceptin.jpg}
            \caption{Modelization of the trastuzumab molecule}
            \label{fig:trastuzumab}
    \end{minipage}\hfill
    \begin{minipage}{0.495\textwidth}
            \centering
            \includegraphics[width=\textwidth]{../Images/trastuzumab.jpg}
            \caption{Trastuzumab is sold by Roche under the name Herceptin
            \cite{roche_herceptin_nodate}}
            \label{fig:herceptin}
    \end{minipage}
\end{figure}

Breast cancer is one of the most common cancers in the world.
It concerns 25\% of cancer cases in women, with around 2 million new cases
each year. This type of cancer has thus been the target of many research efforts.

For a particular type of breast cancers called \emph{HER2 positive}, be it early 
or metastatic, which can be particularly aggressive, multiple treatments can
be considered. Surgery, radiation therapy can be used to remove part of the tumor ;
chemotherapy aims at treating chemically the cancer. However, these treatments are
not always fully effective at removing the tumor, especially for metastatic cancers,
which is were tumor cells travel in the bloodstream and fix themselves in another
part of the body.

Trastuzumab, a humanized monoclonal antibody approved by the FDA in the 
United States in 1998, binds to a transmembrane protein called 
the \emph{HER2 receptor} \cite{zhao_trastuzumab_2021}. This protein is
overexpressed in 15 to 25 percent of breast cancers ; this overexpression
is linked with aggressive forms of the disease \cite{piccart-gebhart_trastuzumab_2005}.

Indeed, the proliferation of this protein at the surface of cells leads to their
uncontrolled multiplication, leading to the formation of a tumor.
Trastuzumab, by binding with the HER2 receptor, prevents its dimerization and
proliferation, and thus stops the growth of the cell.

It can by covalently associated with a cytotoxic called DM1 to form
\emph{trastuzumab emtansine}. This antibody-drug conjugate (ADC) 
is associated with a higher disease-free survival rate
\cite{lambert_ado-trastuzumab_2014}.

Herceptin is delivered by injection over the course of one year.
It costs $17000$ € per patient. Herceptin can also be prescribed for treatment of stomach cancer. 
As this drug is comparatively old, multiple
biosimilars have been developed by concurrent laboratories. This was in the
top-3 selling drugs for Roche Pharma ; it brought a $6,08$ billion dollar
revenue in 2019 alone to the lab \cite{fierce_pharma_herceptin_2020}.


\subsubsection{Adalimumab against autoimmune diseases}

\begin{figure}[H]
    \begin{minipage}{0.495\textwidth}
            \centering
            \includegraphics[width=0.6\textwidth]{../Images/TNFa_Crystal_Structure.rsh.png}
            \caption{Crystal structure of the TNF-$\alpha$ protein}
            \label{fig:TNFa_Crystal_Structure}
    \end{minipage}\hfill
    \begin{minipage}{0.495\textwidth}
            \centering
            \includegraphics[width=\textwidth]{../Images/humira.jpg}   
            \caption{Adalimumab is sold by Abbvie under the name Humira \cite{noauthor_humira_nodate-1}}
            \label{fig:humira}
    \end{minipage}
\end{figure}

The \emph{Tumor Necrosis Factor-alpha} (TNF-$\alpha$) is a transmembrane protein
associated with the mechanism of inflammation. It helps recruiting lymphocytes
and macrophages to help locally fight pathogens.

However, it can be implicated in a number of autoimmune diseases including
ulcerative colitis, Crohn’s disease, plaque psoriasis, and arthritis
\cite{noauthor_adalimumab_2021}.

Adalimumab is a fully human monoclonal antibody approved by the FDA in 2002.
This IgG$_1$ can bind to TNF-$\alpha$ and inhibit its activity, thus reducing the 
signs and symptoms in the diseases cited above, notably for rheumatoid arthritis
\cite{mease_adalimumab_2007}. It is delivered by subcutaneous injection, and mostly
prescribed to patients who have not responded well to previous treatments.

Humira has been for ten year the world top-selling drug \cite{fierce_pharma_humira_2021}.
Being prescribed for treatment of rheumatoid arthritis, psoriatic arthritis and Crohn’s disease,
it grossed $20,39$ billion dollars in sales in 2020.



    \subsection{Market analysis}
  
  
  \section*{Conclusion}

\end{document}
